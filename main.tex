% Copyright (c) 2022 by Lars Spreng
% This work is licensed under the Creative Commons Attribution 4.0 International License. 
% To view a copy of this license, visit http://creativecommons.org/licenses/by/4.0/ or send a letter to Creative Commons, PO Box 1866, Mountain View, CA 94042, USA.

%~~~~~~~~~~~~~~~~~~~~~~~~~~~~~~~~~~~~~~~~~~~~~~~~~~~~~~~~~~~~~~~~~~~~~~~~~~~~~~
% You can add your packages and commands to the loadslides.tex file. 
% The files in the folder "styles" can be modified to change the layout and design of your slides.
% I have included examples on how to use the template below. 
% Some of it these examples are taken from the Metropolis template.
%~~~~~~~~~~~~~~~~~~~~~~~~~~~~~~~~~~~~~~~~~~~~~~~~~~~~~~~~~~~~~~~~~~~~~~~~~~~~~~


\documentclass[
11pt,notheorems,hyperref={pdfauthor=whatever}
]{beamer}

\input{loadslides.tex} % Loads packages and some defined commands

\title[
% Text entered here will appear in the bottom middle
]{Undecidability of First-Order Logic}

\subtitle{Solution to Entscheidungsproblem}

\author[
% Text entered here will appear in the bottom left corner
]{
    Arunachalaeshwaran V R,
    Alan Jojo
}

\institute{
    Indian Institute of Science}
\date{\today}

\begin{document}

% Generate title page
{
\setbeamertemplate{footline}{} 
\begin{frame}
  \titlepage
\end{frame}
}
\addtocounter{framenumber}{-1}

% You can declare different parts as a parentof sections
\begin{frame}{Part I: Demo Presentation Part}
    \tableofcontents[part=1]
\end{frame}
\begin{frame}{Part II: Demo Presentation Part 2}
    \tableofcontents[part=2]
\end{frame}

\makepart{Demo Part}

\section{Introduction}
\begin{frame}
\begin{itemize}
    \item Dummy text \cite{example_2022}
\end{itemize}     
\end{frame}


\begin{frame}[allowframebreaks]{References}
    \printbibliography
\end{frame}
\end{document}