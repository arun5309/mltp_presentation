% Copyright (c) 2022 by Lars Spreng
% This work is licensed under the Creative Commons Attribution 4.0 International License. 
% To view a copy of this license, visit http://creativecommons.org/licenses/by/4.0/ or send a letter to Creative Commons, PO Box 1866, Mountain View, CA 94042, USA.

%~~~~~~~~~~~~~~~~~~~~~~~~~~~~~~~~~~~~~~~~~~~~~~~~~~~~~~~~~~~~~~~~~~~~~~~~~~~~~~
% You can add your packages and commands to the loadslides.tex file. 
% The files in the folder "styles" can be modified to change the layout and design of your slides.
% I have included examples on how to use the template below. 
% Some of it these examples are taken from the Metropolis template.
%~~~~~~~~~~~~~~~~~~~~~~~~~~~~~~~~~~~~~~~~~~~~~~~~~~~~~~~~~~~~~~~~~~~~~~~~~~~~~~


\documentclass[
11pt,notheorems,hyperref={pdfauthor=whatever}
]{beamer}


% Copyright (c) 2022 by Lars Spreng
% This work is licensed under the Creative Commons Attribution 4.0 International License. 
% To view a copy of this license, visit http://creativecommons.org/licenses/by/4.0/ or send a letter to Creative Commons, PO Box 1866, Mountain View, CA 94042, USA.

%~~~~~~~~~~~~~~~~~~~~~~~~~~~~~~~~~~~~~~~~~~~~~~~~~~~~~~~~~~~~~~~~~~~~~~~~~~~~~~
% Add your packages and commands to this file
%~~~~~~~~~~~~~~~~~~~~~~~~~~~~~~~~~~~~~~~~~~~~~~~~~~~~~~~~~~~~~~~~~~~~~~~~~~~~~~

%~~~~~~~~~~~~~~~~~~~~~~~~~~~~~~~~~~~~~~~~~~~~~~~~~~~~~~~~~~~~~~~~~~~~~~~~~~~~~~
\RequirePackage{palatino}
\RequirePackage[utf8]{inputenc}
\RequirePackage[T1]{fontenc}

\usefonttheme{serif}

\usepackage{styles/elegantmacros}
\usefolder{styles}
\usetheme[style=blue]{elegant}

\newcommand{\makepart}[1]{ % For convenience
\part{#1} \frame{\partpage}
}

%~~~~~~~~~~~~~~~~~~~~~~~~~~~~~~~~~~~~~~~~~~~~~~~~~~~~~~~~~~~~~~~~~~~~~~~~~~~~~~

%~~~~~~~~~~~~~~~~~~~~~~~~~~~~~~~~~~~~~~~~~~~~~~~~~~~~~~~~~~~~~~~~~~~~~~~~~~~~~~
% Figures
\RequirePackage{booktabs}
\RequirePackage{colortbl}
\RequirePackage{ragged2e}
\RequirePackage{schemabloc}
%\RequirePackage{natbib}
\RequirePackage{caption}
\RequirePackage{subcaption}
\RequirePackage{tabularx}
\RequirePackage{array}
\RequirePackage{multirow}
\usepackage[
  style=alphabetic, 
]{biblatex}
\addbibresource{references.bib}
\newcolumntype{Y}{>{\centering\arraybackslash}X}

%~~~~~~~~~~~~~~~~~~~~~~~~~~~~~~~~~~~~~~~~~~~~~~~~~~~~~~~~~~~~~~~~~~~~~~~~~~~~~~

%~~~~~~~~~~~~~~~~~~~~~~~~~~~~~~~~~~~~~~~~~~~~~~~~~~~~~~~~~~~~~~~~~~~~~~~~~~~~~~
% Figures
\RequirePackage{wrapfig}
\RequirePackage{pgfplots}
\RequirePackage{graphicx}
\RequirePackage{adjustbox}
\RequirePackage{environ}
\pgfplotsset{compat=1.18}

\makeatletter
\newsavebox{\measure@tikzpicture}
\NewEnviron{scaletikzpicturetowidth}[1]{%
  \def\tikz@width{#1}%
  \def\tikzscale{1}\begin{lrbox}{\measure@tikzpicture}%
  \BODY
  \end{lrbox}%
  \pgfmathparse{#1/\wd\measure@tikzpicture}%
  \edef\tikzscale{\pgfmathresult}%
  \BODY
}
\makeatother
%~~~~~~~~~~~~~~~~~~~~~~~~~~~~~~~~~~~~~~~~~~~~~~~~~~~~~~~~~~~~~~~~~~~~~~~~~~~~~~

%~~~~~~~~~~~~~~~~~~~~~~~~~~~~~~~~~~~~~~~~~~~~~~~~~~~~~~~~~~~~~~~~~~~~~~~~~~~~~~
% Maths 
\RequirePackage{textcomp}
\RequirePackage{amsmath} 
\RequirePackage{amsthm}
\RequirePackage{mathtools}
%\RequirePackage{bbm}
%\RequirePackage{algorithm}
%\RequirePackage[osf,sc]{mathpazo}
%\RequirePackage{pifont}
%\newcommand{\xmark}{\ding{55}}%
%\numberwithin{equation}{section}
\DeclareMathOperator*{\argmax}{arg\,max}
\DeclareMathOperator*{\argmin}{arg\,min}

\setbeamertemplate{theorems}[numbered] % to number

\theoremstyle{definition}
\newtheorem{fact}{Fact}[section]
\newtheorem{examp}{Example}[section]

\theoremstyle{plain}
\newtheorem{definition}{Definition}[section]
\newtheorem{proposition}{Proposition}
\newtheorem{theorem}{Theorem}
\newtheorem{assumption}{Assumption}

\providecommand{\H}{\mathscr{H}}      
\providecommand{\E}{\mathbb{E}}
\makeatletter
\def\munderbar#1{\underline{\sbox\tw@{$#1$}\dp\tw@\z@\box\tw@}}
\makeatother

%~~~~~~~~~~~~~~~~~~~~~~~~~~~~~~~~~~~~~~~~~~~~~~~~~~~~~~~~~~~~~~~~~~~~~~~~~~~~~~
 % Loads packages and some defined commands

\title[
% Text entered here will appear in the bottom middle
]{Undecidability of First-Order Logic}

\subtitle{Solution to Entscheidungsproblem}

\author[
% Text entered here will appear in the bottom left corner
]{
    Arunachalaeshwaran V R,
    Alan Jojo
}

\institute{
    Indian Institute of Science}
\date{\today}

\begin{document}

% Generate title page
{
\setbeamertemplate{footline}{} 
\begin{frame}
  \titlepage
\end{frame}
}
\addtocounter{framenumber}{-1}

% You can declare different parts as a parentof sections
\begin{frame}{Part I: Introduction}
    \tableofcontents[part=1]
\end{frame}
\begin{frame}{Part II: Preliminaries}
    \tableofcontents[part=2]
\end{frame}
\begin{frame}{Part III: Halting Problem}
    \tableofcontents[part=3]
\end{frame}
\begin{frame}{Part IV: Proof of Undecidability of First-order Logic}
    \tableofcontents[part=4]
\end{frame}

\makepart{Introduction}

\section{Motivation}
\begin{frame}
\begin{itemize}
    \item Can we have a machine that can check the proof of any mathematical statement?
    \item Can we have a machine that can prove or disprove any mathematical statement?
    \item Relationship to Godel's incompleteness theorem (contrast with Godel's completeness theorems. Especially different notions of completeness.).
    \item Primary reference: \cite{boolos2002computability}
\end{itemize}     
\end{frame}

\section{Capturing Computation}
\begin{frame}
\begin{itemize}
    \item Turing Computability (captures the notion of algorithm/effective calculability) and relationship to formal languages
    \item Church-Turing Thesis: TODO
    \item Equivalent notions of computability: General recursive functions, Lambda calculus, Cyclic tag systems, Rule 110, most programming languages (might be helpful to think in terms of a computer program in a particular programming language) etc.
    \item Universal Turing Machines (crucially exploits the fact that description of a particular Turing machine is of finite size and hence be encoded as a natural number)
\end{itemize}     
\end{frame}

\section{Undecidability}
\begin{frame}
\begin{itemize}
    \item Not all formal languages can be decided by a Turing machines. Examples are Halting problem, determining validity of a first-order logic formula (Church's theorem), post-correspondence problem, Hilbert's $10^{\text{th}}$ problem, etc.
    \item Existence of such languages can be seen by a simple counting argument.
    \item Existence of undecidable languages implies that certain functions are not computable. Examples are busy beaver function, computing a solution to a set of Diophantine equations (output $\bot$ if no solutions exist), come up with a proof for a first-order logic formula (output $\bot$ if no proof exists) etc.
    \item Notion of recursive enumerability (relationship to Godel's incompleteness theorem. Use Lindenbaum extension to show that such subsets exist.).
\end{itemize}     
\end{frame}

\section{Proof sketch of Church's theorem}
\begin{frame}
    \begin{enumerate}
        \item Consider a particular Turing machine and an unary encoding of its input.
        \item Encode the Turing machine and its input as a first-order logic formula in a manner such that the Turing machine halts iff first-order logic formula is valid (in fact we will be dealing specifically with implication formulas).
        \item For the sake of contradiction assume that there exists an algorithm to decide validity of any first-order logic formula.
        \item Use the previous algorithm to construct an algorithm for the Halting problem.
        \item As Halting problem is undecidable we have a contradiction.
        \item Hence, we conclude that no algorithm can decide the validity of any first-order logic formula.
    \end{enumerate}
\end{frame}

\makepart{Preliminaries}

\section{Formal Languages}
\begin{frame}
\end{frame}

\subsection{Complement of a language}
\begin{frame}
\end{frame}

\section{Turing Machine}
\begin{frame}
\end{frame}

\subsection{Decider vs Enumerator (undecidable vs unrecognizable)}
\begin{frame}
\end{frame}

\subsection{Encoding a Turing Machine as a Finite String}
\begin{frame}
\end{frame}

\subsection{Universal Turing Machine}
\begin{frame}
\end{frame}

\section{Robinson Arithmetic}
\begin{frame}
\end{frame}

\section{Godel's Incompleteness Theorem}
\begin{frame}
\end{frame}

\makepart{Halting Problem}

\section{Statement of Halting Problem}
\begin{frame}
\end{frame}

\section{Proof of Undecidability of Halting Problem}
\begin{frame}
\end{frame}

\makepart{Proof of Undecidability of First-Order Logic}

\section{Intuitive Explanation of the Logic's Signature}
% TODO: In each subsection explain one component of the signature
\begin{frame}
\end{frame}

\section{Intuitive Explanation of the Formula}
% TODO: In each subsection explain one formula
\begin{frame}
\end{frame}

\section{Completing the Proof}
\begin{frame}
\end{frame}

\begin{frame}[allowframebreaks]{References}
    \printbibliography
\end{frame}
\end{document}